\documentclass[a4paper, 12pt]{article}

\usepackage[english,russian]{babel} 
\usepackage[utf8]{inputenc}

\title{Контрольное задание № 1}
\author{Евгений Деин}

\begin{document}
	\maketitle
	Когда слухи об этом дошли до Киева и богослов Халява услышал наконец о такой
	участи философа Хомы, то предался целый час раздумью. С ним в продолжение того
	времени произошли большие перемены. Счастие ему улыбнулось: по окончании курса наук
	его сделали звонарем самой высокой колокольни, и он всегда почти являлся с разбитым
	носом, потому что деревянная лестница на колокольню была чрезвычайно безалаберно
	сделана.
	
	– Ты слышал, что случилось с Хомою? – сказал, подошедши к нему, Тиберий Горобець,
	который в то время был уже философ и носил свежие усы.
	
	– Так ему бог дал, – сказал звонарь Халява. – Пойдем в шинок да помянем его душу!
	
	Молодой философ, который с жаром энтузиаста начал пользоваться своими правами,
	так что на нем и шаровары, и сюртук, и даже шапка отзывались спиртом и табачными
	корешками, в ту же минуту изъявил готовность.
	
	– Славный был человек Хома! – сказал звонарь, когда хромой шинкарь поставил перед
	ним третью кружку. – Знатный был человек! А пропал ни за что.
	
	– А я знаю, почему пропал он: оттого, что побоялся. А если бы не боялся, то бы ведьма
	ничего не могла с ним сделать. Нужно только, перекрестившись, плюнуть на самый хвост ей,
	то и ничего не будет. Я знаю уже все это. Ведь у нас в Киеве все бабы, которые сидят на
	базаре, – все ведьмы.
	
	На это звонарь кивнул головою в знак согласия. Но, заметивши, что язык его не мог
	произнести ни одного слова, он осторожно встал из-за стола и, пошатываясь на обе стороны,
	пошел спрятаться в самое отдаленное место в бурьяне. Причем не позабыл, по прежней
	привычке своей, утащить старую подошву от сапога, валявшуюся на лавке.
\end{document}
