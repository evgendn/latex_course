\documentclass[a4paper,12pt]{article}

\usepackage{cmap} % поиск и копирование в PDF документах
\usepackage[T2A]{fontenc}
\usepackage[utf8]{inputenc}
\usepackage[russian,english]{babel}
\usepackage{amsmath}
\usepackage{alphalph} % для нумерации уравнений буквами

\title{Контрольное задание №2}
\author{Евгений Деин}

\begin{document}
  \maketitle
  % стр. 13 - С раздела 1. Поверхность...  до рис. 3
  \section{ПОВЕРХНОСТЬ В ОБЪЕМЛЮЩЕМ ЕЕ ПРОСТРАНСТВЕ}
  \subsection{Задание поверхности}
  Поверхность может быть задана уравнением одного из типов:
  \begin{equation}
    \label{eq:zl}
    z=f(x,y)
  \end{equation}

  \begin{equation}  
    \label{eq:fl}
    F(x,y,z)=0
  \end{equation}

	которые определяют координату $z$ как явную \eqref{eq:zl} или 	неявную \eqref{eq:fl} функцию координат $x$, $y$. 
	
	%стр. 14 - С первой строки, до раздела 1.2. Касательная плоскость...
	Не всякая поверхность полностью может быть задана при помощи уравнения типа \eqref{eq:zl}, так как в ряде случаев одной комбинации значений $x$ и $y$ соответствует не единственная точка на поверхности (например, рис.3,б). В подобных случаях приходится либо прибегать к заданию поверхности неявным уравнением типа \eqref{eq:fl}, либо использовать уравнение типа \eqref{eq:zl}, но не для всей поверхности сразу,
	а по частям.

	Уравнение типа \eqref{eq:fl} для сферы, изображенной на рис. 3, б, имеет вид
	\[x^{2} + y^{2} + (z-c)^{2} = R^{2}.\]
	Та же сфера может быть задана и уравнением типа \eqref{eq:zl}, но по частям:часть поверхности, лежащая ниже плоскости $z$=$c$, задается уравнением
	\begin{equation}
	\label{eq:z3l}
	z=c-\sqrt{R^{2} - (x^{2} + y^{2})},
	\end{equation}
	а находящаяся выше плоскости $z = c$ — уравнением
	\begin{equation}
		\label{eq:z4l}
		z=c+\sqrt{R^{2} - (x^{2} + y^{2})};
	\end{equation}
	точки поверхности, лежащие в плоскости $z = c$, могут быть получены как из \eqref{eq:z3l}, так и из \eqref{eq:z4l}.
	
	Заметим, что невозможность представления всей поверхности полностью в форме \eqref{eq:zl} в некоторой системе координат не исключает представимости уравнения этой же поверхности в такой форме в другой системе координат.

	%стр. 15 - Со слов "Кривизна ka выражается...", до конца страницы (без сноски).
	\subsection{Касательная плоскость. Нормальные сечения}
  	Кривизна ka выражается следующей формулой$^1$ (см. рис. 6):
  	\begin{displaymath}
  		k_a=\lim_{l \to 0}
  		\frac{2h}{l^{2}} =
  		\lim_{l \to 0}
  		\frac{2f(x,y)}{l^2},
	\end{displaymath}  
	
	% стр. 16 - Со слов "Имея ввиду, что при...", до конца страницы (без сноски).
	Имея в виду, что при $x \rightarrow 0$ и $y \rightarrow 0$ величина $\epsilon$ также устремляется к нулю, получаем:		
	\begin{equation}
	\label{eq:z5l}
	\begin{split}
		k_a=\lim_{l \to 0}
		\frac{f_{xx}(0,0)x^2 + 2f_{xy}(0,0)xy +f_{yy}(0,0)y^2 + 2\epsilon l^2}{l^2} = \\
		=f_{xx}(0,0)\cos^2{\alpha} + f_{xx}(0,0)\sin^2{\alpha} + 2f_{xy}(0,0)\sin{\alpha}\cos{\alpha} 
	\end{split}
	\end{equation}
	
	Здесь учтено, что
	\begin{displaymath}
		\frac{x}{l}=\cos{\alpha};
		\frac{y}{l}=\sin{\alpha}.
	\end{displaymath}
	В частности, 
	\begin{equation}
		\label{eq:z6l}
		k_x=f_{xx}(0,0);
		k_y=f_{yy}(0,0).
	\end{equation}
	
	Таким образом, зная кривизны любых двух ортогональных
	нормальных сечений в точке $A$, например $k_x$ и $k_y$, а также зная $f_{xy}(0,0)$, можно найти кривизну любого другого нормального сечения в той же точке.

	% стр. 17 - Со слов "Формулы преобразования...", до конца страницы.
	\subsection{Кривизна нормальных сечений поверхности — тензор второго ранга}
	Формулы преобразования компонентов симметричного тензора второго ранга имеют вид:
	\begin{displaymath}
		\left. \begin{aligned}
		a_{x_{1}'x_{1}'}=a_{x_1 x_1}l_1^2 + a_{x_2 x_2}m_{1}^2 +& 2a_{x_1 x_2}l_1 m_1 = a_{x_1 x_1}\cos^2{\alpha} + a_{x_2 x_2}\sin^2{\alpha} + \\ +&
		2a_{x_1 x_2}\cos{\alpha}\sin{\alpha} \\
		a_{x_{2}'x_{2}'}=a_{x_1 x_1}l_2^2 + a_{x_2 x_3}m_{2}^2 +& 2a_{x_1 x_2}l_2 m_2 = a_{x_1 x_1}\sin^2{\alpha} + a_{x_2 x_2}\cos^2{\alpha} - \\ -&
		2a_{x_1 x_2}\cos{\alpha}\sin{\alpha} \\
		a_{x_{1}'x_{3}'}=a_{x_3 x_1}l^2=a_{x_2 x_2}l_1 l_2 +& a_{x_2 x_2}m_1 m_2 + a_{x_1 x_1}(l_1 m_2 + l_2 m_1) =  \\ =
		-a_{x_1 x_1}\cos{\alpha}\sin{\alpha} +& a_{x_2 x_2}\sin{\alpha}\cos{\alpha} + a_{x_1 x_1}(\cos^2{\alpha} - \sin^2{\alpha}) = \\ = 
		\frac{1}{2}(a_{x_2 x_2} - a_{x_1 x_1})&\sin{2\alpha} + a_{x_1 x_1}\cos{2\alpha}.
		\end{aligned} \right\}
	\end{displaymath}
	
	% стр. 19 - С начала страницы,  до слов "Для получения выражения..."
	Нетрудно видеть, что формула (10) действительно справедлива. С целью получения выражения для $f_{x_1 y_1}(0,0)$ применим правило дифференцирования сложной функции:
	\begin{equation}
	\begin{gathered}
		f_{x_1 y_1}=\frac{\partial^2 f}{\partial x_1 \partial y_1}=\frac{\partial}{\partial x_1}\left(\frac{\partial f}{\partial y_1} \right)=\frac{]\partial}{\partial x_1}\left(\frac{\partial f}{\partial x}\frac{\partial x}{\partial y_1} + \frac{\partial f}{\partial y}\frac{\partial y}{\partial y_1} \right) = \\ = 
		\frac{\partial}{\partial x} \left( \frac{\partial f}{\partial x}\frac{\partial x}{\partial y_1} \right)\frac{\partial x}{\partial x_1} +  \frac{\partial}{\partial y} \left( \frac{\partial f}{\partial x}\frac{\partial x}{\partial y_1} \right)  \frac{\partial y}{\partial x_1} + \frac{\partial}{\partial y} \left( \frac{\partial f}{\partial x}\frac{\partial x}{\partial y_1} \right)  \frac{\partial y}{\partial x_1} + \\ +
		\frac{\partial}{\partial y} \left( \frac{\partial f}{\partial x}\frac{\partial x}{\partial y_1} \right)  \frac{\partial y}{\partial x_1} = \frac{\partial^2 f}{\partial x^2} \frac{\partial x}{\partial y_1} \frac{\partial x}{\partial x_1} + \frac{\partial^2 f}{\partial x \partial y} \frac{\partial x}{\partial y_1} \frac{\partial x}{\partial x_1} + \\ +
		\frac{\partial^2 f}{\partial x \partial y} \frac{\partial x}{\partial y_1} \frac{\partial x}{\partial x_1} + \frac{\partial^2 f}{\partial y^2} \frac{\partial x}{\partial y_1} \frac{\partial x}{\partial x_1}.
	\end{gathered}
	\end{equation}
	
	Здесь как функция $f_{x_1 y_1}$ так и производные функции $f$ в правой части равенства рассматриваются при $x=0$, $y=0$.

	% стр. 28 - Раздел 3.2 полностью
	\section{НЕКОТОРЫЕ ФОРМУЛЫ И ТЕОРЕМЫ}
	\subsection{Векторное уравнение поверхности}
	Положение точки на поверхности определяется координатами $\alpha_1$ и $\alpha_2$, эта же точка по-другому может быть задана при помощи \textit{радиуса-вектора} \textbf{r} , имеющего неподвижное начало в некоторой точке пространства и конец в точке поверхности ($\alpha_1, \alpha_2$). Очевидно, что \textbf{r} является функцией от $\alpha_1$ и $\alpha_2$:
	\begin{equation}
		\label{eq:z8l}
		\textbf{r} = \textbf{r}(\alpha_1, \alpha_2)
	\end{equation}
	
	Векторному равенству \eqref{eq:z8l} соответствуют три скалярных равенства:
	\begin{displaymath}
		x=x(\alpha_1, \alpha_2); y=y(\alpha_1, \alpha_2); z=z(\alpha_1, \alpha_2).
	\end{displaymath}
\end{document}	
